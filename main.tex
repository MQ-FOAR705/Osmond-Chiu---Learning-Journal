\documentclass{article}
\usepackage{blindtext}
\usepackage{titlesec}
\usepackage[utf8]{inputenc}
\usepackage{hyperref}
\usepackage{lipsum}
\setlength{\parindent}{0em}
\setlength{\parskip}{1em}


\title{FOAR705 Learning Journal}
\author{Osmond Chiu}
\date{Semester 2, 2019}
\begin{document}
\maketitle
\tableofcontents
\newpage

\section{List of Useful Errors}\par

CSV Exports - \autoref{sec: CSVexport}\par
Customised Lists - \autoref{sec: Customlist}\par
Underscoring when using bullet points - \autoref{sec: Underscore}\par
Not executing Commands in LaTex - \autoref{sec: Commands}\par
Paragraphs - \autoref{sec: Paragraphs}\par

\newpage
\section{Data Carpentry}
\section*{9/8/2019 - 7:10pm}
I’ve decided to get a head start by reading instructions for the first exercise ‘Formatting data tables in Spreadsheets’ and have read the Data Organization in Spreadsheet lessons.\par
At this stage, from a quick glance and the explanations it seems relatively straight forward but there might be issues that come up.\par
\section{Data Cleaning Exercise}
\section*{10/8/2019 – 9:15am}
Objective:\par
Identify errors in SAFI messy.xlsx spreadsheet\par
Action:
\begin{itemize}
\item Downloaded SAFI messy copy.xlsx
\item Opened ‘SAFI messy.xlsx’ and saved a copy of the file named ‘SAFI messy copy.xlsx’
\item Opened up and examining ‘SAFI messy copy.xlsx
\end{itemize}
It is clear there are a number of things that need to be cleaned up. These include:\par
\begin{itemize}
\item Problematic field names with spaces between column titles e.g. wall type, floor type, water use
\item Attempted null values that will not be recognised e.g. -999 and -99 
Not filling in zeros and leaving them blank instead
\item Use of yellow highlighting and text in cells to provide comments, not data
\item Merged title cells
\item Multiple bits of information in the same column e.g. livestock owned and number column
\item Inconsistent use of numbers versus yes and no in cases e.g. ‘water use’ column
\item Multiple tables in the same sheet
Using special characters like *
\item Incorrect data in ‘livestock owned and numbers’ column and in the columns under (livestock)
\end{itemize}
Result: Identified a range of errors that needed to be cleaned up.\par
It will require recoding some variables, renaming case titles, the creation of new cases and the removal of formatting in a new copy of the spreadsheet. The original raw data should not be changed. An explanation about how and why incorrect data is recoded should be kept to be made clear.\par
Objective:\par 
Identify errors in ‘Mozambique’ sheet to clean up\par
Action: 
\begin{itemize}
\item For the Mozambique sheet, the following actions would need to occur
\item Rename ‘wall type’ to be ‘wall type’
\item Rename ‘floor type’ to be ‘floor type’
\item Recode ‘mabati sloping’ as ‘mabatissloping’ in wall type column
\item Recode ‘eerth’ as ‘earth’ in floor type column
\item Recode ‘-99’ as blank
\item Add column named ‘barn’ next to ‘rooms’ column
\item Code ‘yes’ for row ‘key id = 10’ in barn column
\item Code ‘no’ for rows ‘key id = 1 to 9’
\item Add column named ‘cowshed‘ to the right of ‘barn’
\item Code ‘no’ for all rows in ‘cowshed‘
\item Remove yellow highlighting from ‘rooms’ column
\item Delete ‘include barns’ text and remove yellow highlighting next to it
\item Create a ‘oxen’ column to the left of poultry
\item Create ‘cows’, ‘goats’ and ‘total’ column to the right of ‘poultry’ column
\item Recode ‘key id = 2’ row in ‘livestock owned and numbers’ column as ‘1, (poultry)’
\item Recode ‘key id = 6’ row in ‘livestock owned and numbers’ column as ‘0, (none)’
\item Code ‘oxen’, ‘cows’, ‘goats’ and ‘total’ column with data from ‘livestock owned and numbers’ column using 1 or 0
\item Recode ‘yes’ in ‘poultry’ column as 1
\item Recode ‘no’ in ‘poultry’ column as 0
\item Delete ‘livestock owned and numbers’ column
\item Recode ‘-999’ in ‘plots’ column as a blank
\item Rename ‘water use’ column as ‘water use’
\item Create ‘only summer use’ column next to ‘water use’
\item Code ‘yes’ in ‘water use’ column for ‘key id = 2’
\item Code ‘no’ in ‘water use’ column for remaining rows
\item Recode ‘N’ as ‘no’ in ‘water use’ column
\item Recode ‘Y’ as ‘yes’ in ‘water use’ column
\item Recode ‘1’ as ‘yes’ in ‘water use’ column
\item Move columns under (livestock) next to dwelling then plots so they line up
\item Delete merged cells ‘livestock’, ‘plots’ and ‘dwelling’ which are used for titles of tables
\item Create a column named ‘Country’ next to ‘key id’ columns
\item Code all rows in ‘Country’ as ‘Mozambique’
\item Create column ‘cow died in april not replaced’ next to ‘look after cows’
\item Code ‘no’ in ‘cow died in april not replaced’ for remaining unfilled rows
\end{itemize}
Error: None\par
Result: Errors identified for cleaning up in 'Mozambique' sheet \par
Objective: \par
Identify errors in ‘Tanzania ’ sheet to clean up\par
Action:\par
For the Tanzania sheet:
\begin{itemize}
\item Rename ‘wall type’ to be ‘wall type’
\item Rename ‘floor type’ to be ‘floor type’
\item Rename ‘Look after cows’ to be ‘look after cows’
\item Create column named ‘barn’ to the right of ‘rooms’
\item Code ‘no’ in all rows for ‘barn’
\item Create column named ‘cowshed’ to the right of ‘barn’
\item Code ‘yes’ in ‘cowshed’ for ‘key id = 8’
\item Code ‘no’ in ‘cowshed’ for remaining unfilled rows
\item Recode 4* as 4 in ‘rooms’
\item Delete text ‘* = includes Cowshed’
\item Create column ‘cow died in april not replaced’ next to ‘look after cows’
\item Code ‘yes’ in ‘cow died in april not replaced’ for row ‘key id = 3’
\item Code ‘no’ in ‘cow died in april not replaced’ for remaining unfilled rows
\item Recode ‘Yes/no*’ in ‘look after cows’ as ‘Yes’
\item Delete text ‘* Cow died in April and not replaced’
\item Recode 0 in ‘Oxen’ at ‘key id = 8’ row as 1
\item Recode 0 in ‘Cows’ at ‘key id = 3’ row as 1
\item Recode 2 in ‘Poultry’ at ‘key id = 2’ row as 1
\item Recode 4 in ‘Total’ at ‘key id = 6’ row as 3
\item Recode ‘Yes’ in ‘key id = 5’ row as 1
\item Recode ‘No’ in ‘key id = 5’ row as 0
\item Recode blanks in ‘Oxen’, ‘Poultry’, ‘Goats’ and ‘Cows’ as 0
\item Move columns under (livestock) next to dwelling so they line up
\item Delete merged cell ‘Dwelling’
\item Delete text ‘(Livestock)
\item Create a column named ‘Country’ next to ‘key id’ columns
\item Code all rows in ‘Country’ as ‘Tanzania
\end{itemize}
Error: None \par
Result: Now that the data is cleaned, a new sheet will be created and the data can be copied across with columns aligned 
\section{Metadata Exercise}
\section*{10/8/2019 – 3:06pm}
I have read the second exercise on Metadata and think I understand what it is asking.
\par
Objective:\par
Complete Metadata Exercise\par
Action:\par
Download ‘SAFI Clean.csv’\par
Opening ‘SAFI Clean.csv’,\par
There are a number of issues with the data that would be clarified with better metadata. There are a number of things that are immediately not obvious. Some of the questions I would have include:
\begin{itemize}
\item What is key ID referring to? Is it to a household or family?
\item Does 'no membrs' mean how many people there are in household?
\item What does ‘years liv’ mean? Does it mean lived in the specific house or village?
\item What does ‘mem assoc’ mean?
\item What does ‘affect conflicts’ mean? Does it relate to any other question?
\item What does ‘liv count’ mean?
\item Does ‘no meals’ mean number of meals or no meals? If the former, what frequency?
\item What does 'months lack food' exactly mean? Is it reliant on food from others \item versus being self-sufficient?
\item Is 'items owned' by the household?
\end{itemize} 
The metadata that should be recorded about this dataset includes:
\begin{itemize}
\item The exact wording of questions used in the interviews
\item The full definitions of variables used
\item The full range of answers allowed for questions
\item How the answers were coded if they did not fit into the answers allowed
\item Whether the answer was from an interview or determined by the interviewer
\end{itemize}
Error: None\par
Result: Completed Metadata Exercise\par
I decided to save the solutions to the metadata exercise into a file and to upload it to Github for future reference.\par
Objective: Test uploading to Github\par
Action:\par
\begin{itemize}
\item Copy answers for metadata exercise into a text file
\item Save text file as ‘Osmond Chiu - Formatting data tables in Spreadsheets - Metadata questions.txt’
\item Go to URL https://github.com/MQ-FOAR705/Chiu-Exercises
Click Upload file
Click commit to change
\end{itemize}
Error: None\par
Result: File uploaded.
\section{Dates Exercise}
\section*{16/08/2019 - 3:01pm}
All dates are stored as the number of days in spreadsheet since day X. Dates will often be wrong.\par
Make sure the month is written as letters.\par
Put date format in metadata\par
\section{Data As Dates Exercise}
\section*{16/8/2019 - 6:16pm}
Objective: Separate dates into component\par
Action:
\begin{itemize}
\item Open ‘SAFI Dates.xlsx’
\item Save copy with file name ‘SAFI Dates Exercise Copy’.xlsx
\item Add columns ‘month’, ‘day’, ‘year’  to MMDDYYYY tab
\item Add formulas =Month(A2), =Day(A2), =Year(A2) under columns
\item Copy formula by dragging fill handle for new columns to the last row with data
\end{itemize}
Error: None\par
Result: Dates in A2 column are separated into month, day and year columns\par
Objective:\par
Default year\par
Action:
\begin{itemize}
 \item In file name ‘SAFI Dates Exercise Copy’.xlsx
\item Add columns ‘interview date’  to MMDDYYYY tab
\item Add new value ‘11/17’ to A2 column
\item Copy formula by dragging fill handle for new columns to the last row with data
\end{itemize}
Error: None\par
Result: Dates comes up as 1 November 2017
\section{Data Validation Exercise}
\section*{17/8/2019 - 9:18am}
Objective:\par
Restricting data to numerical values\par
Action:
\begin{itemize}
\item Open ‘SAFI Clean.csv’
\item Save copy ‘SAFI Clean Copy.csv’
\item Select column titled ‘rooms’
\item Click ‘Data’ menu then select ‘Data validation’
\item Select ‘Whole number’ in ‘Allow’
\item Fill in ‘1’ minimum and ‘30’ as the maximum values
\item Click ‘Input Message’ tab
\item Fill in ‘Title’ with ‘Acceptable values’
\item Fill in ‘Input Message’ with ‘The number of rooms must be between 1 and 30’
\item Click ‘Error alert’ tab
\item Click ‘Style’ and choose ‘Stop ’
\item Fill in ‘Title’ with ‘Wrong Figure’
\item Fill in ‘Error Message ’ with ‘The number chosen is not between 1 and 30’
Click ‘Ok’
\end{itemize}
Error: None\par
Result: A yellow box appears when the cursor hovers over the column. The box has a title ‘Acceptable values’ with ‘The number of rooms must be between 1 and 30’ underneath.\par
When a figure larger than 30 is entered into a cell in the column, a pop message titled ‘Wrong Figure’ stating ‘The value to be entered must be a whole number between 1 and 30’ appears.\par
Objective:\par
Restricting data to entries from a list\par
Action:
\begin{itemize}
\item Select column titled ‘village ’
\item Click ‘Data’ menu then select ‘Data validation’
\item Select ‘List’ in ‘Allow’
\item Fill in Source with ‘God’, ‘Ruaca’ and ‘Chirodzo’ as values separated by commas
\item Click ‘Input Message’ tab
\item Fill in ‘Title’ with ‘Acceptable values’
\item Fill in ‘Input Message’ with ‘The villages are God, Ruaca or Chirodzo’
\item Click ‘Error alert’ tab
\item Click ‘Style’ and choose ‘Stop ’
\item Fill in ‘Title’ with ‘Wrong Village ’
\item Fill in ‘Error Message ’ with ‘Choose either God, Ruaca or Chirodzo’
Click ‘Ok’
\end{itemize}
Error: None\par
Results: A yellow box appears when the cursor hovers over the column. The box has a title ‘Acceptable values’ with ‘The villages are God, Ruaca or Chirodzo’ underneath. There is a button that opens a drop down menu of the three options.\par
When a figure larger than 30 is entered into a cell in the column, a pop message titled ‘Wrong Value’ stating ‘Choose either God, Ruaca or Chirodzo’ appears.
\section{Exporting data}\label{sec: CSVexport}
\section*{17/8/2019 - 10:03am}
Objective:\par
Export spreadsheet to .csv\par
Action:
\begin{itemize}
\item Open file name “SAFI dates.xlsx
\item Click ‘File’ then ‘Save As’
\item Choose .csv as ‘Format’ then Click Save
\end{itemize}
Error: This workbook cannot be saved in the selected file format because it contains multiple sheets.\par
Result: Option of saving the ‘Active sheet’ or cancelling the save.\par
If a spreadsheet has multiple sheets as SAFI dates.xlsx did and you want to export it to .csv, you will need to separate the sheets or place the information into the same sheet.\par
Objective:\par
Export MMDDYY sheet to .csv\par
Action:
\begin{itemize}
\item Open file name “SAFI dates.xlsx, choose MMDDYY sheet
\item Click ‘File’ then ‘Save As’
\item Choose .csv as ‘Format’ then Click Save
\item Option of saving the ‘Active sheet’ or cancelling the save.
\item Click ‘Continue’ when the message ‘This workbook contains features that will not work or may be removed if you save it in the selected file format. Do you want to continue?’ pops up
\end{itemize}
Error: None\par
Result: First sheet saved as .csv file\par
Opening the SAFI dates.csv file in a text editor, the data is organised by rows with the data separated by columns.
\section{Table of Errors}\label{sec: Customlist}
\section*{21/8/2019 – 7:05pm}
A conversation on Slack in the \#assignment channel indicated that a table of errors needed to be included in the learning journal and that a cross-reference table of errors should be created.\par
Links were provided to some guidance and code to create a table. The links were \begin{verbatim}
https://texblog.org/2011/09/09/10-ways-to-customize-tocloflot/\end{verbatim} and \begin{verbatim}https://en.wikibooks.org/wiki/LaTeX/Labels_and_Cross-referencing#Sections\end{verbatim}\par
Objective:\par
Create a cross-referenced table of errors\par
Action:\par
\begin{itemize}
    \item Identify errors and label with code and replace X with the useful error name \begin{verbatim} \label{error:X}    \end{verbatim} 
    \item Load the tocloft package by entering \begin{verbatim} \usepackage{tocloft}\end{verbatim} 
    \item Create a list by entering \begin{verbatim}\newcommand{\listerrorname}{Table of Errors}

\newlistof{error}{err}{\listerrorname}
\newcommand{\error}[1]{%
\refstepcounter{error}
  \par\noindent\textbf{error \theerror. #1}
\addcontentsline{err}{error}
{\protect\numberline{\thechapter.\theerror} #1}}\end{verbatim} 
\item Enter the following code \begin{verbatim}
\error{My first error}
\label{err:error1}
\end{verbatim} 
\end{itemize}
Error: Title of list generates but no contents of the list\par
Result: Title of list 'Table of Useful Errors' and 'error 1. My first error' generates but no contents of the list\par
Unclear from the code exactly how to make it work. Given the amount of time it has taken to try to work this out, I will need to just create a Table at the end and come back to cross-referencing later.
\section{Cross-references}
\section*{22/8/2019 - 6:50pm}
On the Slack \#assignment channel, Emily Hunt suggested a hack to get around my problems with attempting to generate a customised list using cross-referencing.\par
Objective: Cross-reference useful errors\par
Action:
\begin{itemize}
\item Loading the hyperref package by using the command \begin{verbatim}\usepackage{hyperref}\end{verbatim}
\item Labelling the useful errors, for example 
\begin{verbatim}\label{sec: CSVexport}\end{verbatim}
\item Cross-references by using the command \begin{verbatim}\autoref{sec: CSVexport}\end{verbatim}
\end{itemize}
Error: None\par
Result: Hyperlinked Section number generated. Additional text explaining where it goes needs to be added.
\newpage
\section{Proof of Concept}
\section*{10/8/2019 – 10:57am}
I’ve started to have a think about the proof of concept for scoping.\par
There are a range of jobs that will need to be done for my thesis, including a fair few tedious and repetitive tasks. Automating some of those processes would free up time and help a lot.\par
Given my thesis is about re-nationalisations in Australia, it will involve a fair bit of trawling for information through Trove, government documents, websites and newspapers. From what I can tell to date, there is not much on re-nationalisation in Australia but plenty on problems with privatisation. Identifying examples of re-nationalisations to examine and available sources would be very useful.\par
There is also the more general tasks of formatting and referencing and task management.
\section{Jobs, Pains and Gains }
\section*{12/8/2019 - 9:55pm}
Reflecting on my scoping project, there are plenty of obvious tasks and pain points for the thesis though the technical solutions are less immediately obvious.\par
Going through the range of tasks that need to be done for my thesis, there are quite a few. These include developing my research question to data collection to data analysis to writing and editing and referencing. It will be time consuming and I will also need to balance it with other work and personal commitments.\par
There are some obvious tools that can be used to ease pain that were mentioned during the class such as LaTex, Github, referencing and project management software.\par
Opportunities for gain would be through automating searches. The limited academic literature on re-nationalisation in Australia means that it would be handy if it could trawl through Trove and other newspaper archives, especially over the last decade. It may be a better source for information than academic journals.
\section{Creating a Document in LaTex}
\section*{14/8/2019 - 7:30pm}
I’ve decided to try to use LaTex for my scoping project. I had initially drafted my scoping project on Word but will copy the text over.\par
I have not used LaTex before but playing around with it, the commands for basic formatting does not look too difficult to remember. I quickly looked at the guides on Overleaf.\par
I created a new blank document then started from scratch by removing the existing code.\par
Objective: Generate Title and author\par
Action:

\begin{verbatim}\title{Proof of Concept Scoping Exercise}|
\author{Osmond Chiu}
\maketitle
\end{verbatim}
\par
Click ‘compile’\par
Error: Compile error\par
Result: No document generated\par
I remembered that I needed to include the type of document and also to say beginning and end.\par
Objective: Generate document with title and author\par
Action:
\begin{verbatim}\documentclass{article}
\title{Proof of Concept Scoping Exercise}
\author{Osmond Chiu}
\begin{document}
\maketitle
\end{document}
\end{verbatim}
Error: None\par
Result: Generated document with title, author and current date\par
I was surprised that the date appeared as I did not include the command for date. I assume this is automatic for the date it was compiled.\par
The next step was organising my contents by creating headings, subheadings and bullet point lists.\par
Objective: Create section and section headings\par
Action:
\begin{verbatim}
\section*{Planned Thesis Topic}
\section*{Tasks to be done for thesis}
\section*{Potential pain points}
\section*{Data Collection}
\section*{Data Analysis}
\section*{Writing}
\section*{Work Commitments}
\section*{Potential pain relievers}
\section*{Managing Tasks and Work Commitments}
\section*{Writing}
\section*{Referencing and formatting}
\section*{Out of scope}
\section*{Opportunities to make gains}
\section*{Data Collection}
\end{verbatim}
Error: None\par
Result: Section and section headings created without numbers\par
Objective: Create itemised lists\par
Action:\par
There are a range of tasks that will need to be done for my thesis. The tasks will include (but not be limited to):
\begin{itemize}
\item Developing a research question and plan;
\item Writing my research proposal;
\item Collecting data;
\item Doing a literature review;
\item Analysing the data;
\item Writing a dissertation plan
\item Writing my first draft;
\item Editing the draft;
\item Proofing; and
\item Referencing and doing a bibliography.
\end{itemize}
Collecting data and doing the literature review will involve (but not be limited to):
\begin{itemize}
\item Compiling a list of examples of re-nationalisation in Australia and finding relevant further background information;
\item Identifying key Australian case studies to explore in further detail including the broader socio-economic context;
\item Collating existing literature on re-nationalisations in Australian and overseas; and
\item Developing questions and conducting interviews with key individuals and organisations involved in Australian case studies.
\end{itemize}
Analysing data may involve (but not be limited to):
\begin{itemize}
\item Coding data on key factors applicable to re-nationalisations to identify any patterns;
\item Applying theoretical frameworks to determine if they provide good explanations for re-nationalisations; and
\item Examining interview transcripts to identify any common themes mentioned by interviewees.
\end{itemize}
Error: None\par
Result: Itemised lists generated\par
The thing that is unclear is how to easy leave a space between lines without needing to specify how large the gap is.
\section{Uploading Scoping Project}
\section*{14/8/2019 - 10:30pm}
After editing the text of my scoping project, it is now ready to be submitted. I will need to download it as a PDF and also in tex format.\par
Objective:\par Download LaTex document as PDF\par
Action:\par
Click ‘recompile’\par
Click Download PDF\par
Error: None\par
Result: Downloaded PDF\par
The downloaded PDF was then submitted via iLearn\par
Objective:\par
Download tex file\par
Action:\par
Click ‘Menu’\par
Under Download, click ‘Source’\par
Error: None\par
Result: No download.\par
It wasn’t clear what happened. I tried it a few times but it did not appear on my system even though it appeared to download. I changed browsers from Safari to Chrome and it seemed to work. Unclear why.\par
The downloaded .tex file was then uploaded to Cloudstor
\section{Revising Scoping Exercise}
\section*{16/08/2019 - 7:01pm}
After re-examining the guidance Week 2 in Cloudstor, I did some additional editing of my scoping exercise to remove references to tools and to focus on the jobs, pains, pain relievers, gains and gain opportunities.\par
The revised version was uploaded by repeating the previous steps I had taken.
\section{Pasting Text into LaTex}
\section*{19/8/2019 - 6:04pm}
I've decided to shift my Learning Journal over from Word to Overleaf. I copied and pasted the text from Word into a blank document. During the editing process I have a few errors, some of which I could resolve, others which I have not resolved yet:\par
\textit{Underscores and bullet points do not work}\label{sec: Underscore}\par
An attempted list does not work and all the bullet points are in a single line. I tried to work out why this error was occurring and realised it was because of the inclusion of underscores. I resolved the issue by replacing the underscores with a space.\par
\textit{Commands}\label{sec: Commands}\par
It was initially unclear how to include commands in LaTex without executing them when I copied across the text for the learning journal. I entered 'how to display latex commands in latex' into Google and an Overleaf FAQ page that suggested using  \begin{verbatim}\begin{verbatim} \end{verbatim} \par
\textit{Paragraphs}\label{sec: Paragraphs}\par
I had initially used  \begin{verbatim}\par\end{verbatim} to create paragraphs but it was not able to create a space between lines. Looking through the FAQ, I found that using the double backslash could do it.\par
Noticing the warnings in the log and output file, after guidance from the lecture, I looked at \begin{verbatim}https://www.overleaf.com/learn/latex/Paragraph_formatting\end{verbatim} and determined I could replace the double backslash with a \begin{verbatim}\par\end{verbatim} and insert the following code \begin{verbatim}\setlength{\parindent}{0em}
\setlength{\parskip}{1em}\end{verbatim}

\textit{Italics}\par
To distinguish the commands to execute, I attempted to italicise slabs. Because of the use of brackets in multiple commands it became very difficult to close the formatting. I gave up for the time being but will reinvestigate.\par
\textit{Table of Contents}\par
I have attempted to create a table of contents using 'Chapters' but it does not seem to be working. As a work around, I have used 'Sections' instead.
\section{Computational thinking}
\section*{19/8/2019 - 6:04pm}
Applying 'computation thinking' to my scoping exercise to break down pains and gains into smaller parts.\par 
The pains are:
\begin{itemize}
\item Research plan
\item Data collection
\item Literature review
\item Coding data
\item Analysing data
\item Writing
\item Editing and proofing
\item Formatting
\item Referencing 
\end{itemize}
The opportunities for gain are:
\begin{itemize}
\item Research plan
\item Data collection
\item Literature review
\item Formatting
\item Referencing
\end{itemize}
What are the patterns in the problems and how can the solutions be revised to produce a step-by-step guide?
\section{Out of Scope}
\section*{21/8/2019 - 9:04pm}
The feedback from my initial scoping exercise was that it did not include what was outside the scope of the project.\par
I identified a few jobs that I stated were pains that were ultimately jobs requiring human judgement. They were 
\begin{itemize}
\item making decisions that influence data collection such as the scope of my research question or definitions of re-nationalisation
\item analysing summarised sources for literature reviews
\item trying to apply broader theoretical frameworks
\item understanding the broader context to case studies
\item involving others in proofing and editing
\end{itemize}
\section{Decomposing Gains}
\section*{21/8/2019 - 9:32pm}
I have examined the Gains I included in my scoping exercise to break them down into smaller more manageable tasks. It appears there will be some overlap with Pains.

\section{Computational Analysis for Data Collection}
\section*{22/8/2019 - 8:01pm}
I have decided that the automation of data collection will be the achievable process for my Proof of Concept.\par
This was on the basis of excluding jobs that involved human judgement and that there may be some more obvious off-the-shelf solutions for other pain relievers and gains. Automating it will have the maximum gain in terms of improved research as well as saving time.\par
I have broken the job down into the following tasks that need to be stepped out:
\begin{itemize}
\item Defining the research aim
\item Determining keywords for research
\item Identifying available sources
\item Researching what tools interact with those sources
\item Testing the tools
\item Saving sources
\item Organising sources
\item Analysing sources
\item Producing metadata
\item Deciding if the process needs to be repeated because the research aim is not met
\end{itemize}

\newpage
\section{Class notes and other homework}
\section*{12/8/2019 - 8:45pm}
\subsection*{Homework Exercise}
We were asked to think of two examples of problem data produced by our discipline. For sociology, the two that come to mind is data from surveys and secondary data sources such as government statistics. I may use neither for my own thesis research proposal so have outlined theoretical issues based on what I know.\par
While surveys can yield significant data, it depends on the exact questions asked and the available answers. Surveys may not capture how people act but rather what they think or believe they should think.\par
Quantitative surveys may not have an option that reflects the interviewees actual response. Qualitative surveys provide more options but bad coding of answers by the interviewer may twist responses into something that do not reflect what was said.\par
Data using secondary sources is often used but there can be problems. Using different datasets that might have different assumptions, for example, trying to use old government statistical data over a long time period when the collection method and calculations have changed.

\section*{16/8/2019 - 2:10pm}
\subsection*{FOAR705 Week 3 Class}
Revised date for scoping study. Scoping study uploaded to Cloudstor can be deleted. Contact Brian on DM to get it deleted from iLearn on Slack.\par
If you don’t define the problem, you won’t know if you have solved it.\par
Scoping is what is the point of the exercise. The aim is to think about the problem without the context of the solution.\par
Stackoverflow aimed at being a site to ask about problem, not solutions.\par
Elaboration is about identifying what technology (tool or technique) has solved the problem e.g. checklist is a technology. The aim is to not invent things from scratch and find how most people have solved a similar problem.\par
Test technology to see if we understood its claims correctly as it can be portrayed with rose coloured glasses.\par
Note that 80-90\% of software products fail.\par
Thesis is a project. If you don’t define the problem the thesis will face, you won’t know.\par
Formatting cannot have meaning.\par
Controlled vocabulary - saying at the start, the only options that can be chosen. Any other value is an error. There is a prepopulated list (allows Other).\par
Each variable should have an explanation in metadata.\par
Metadata is a description of your dataset such that someone will be able to add to it. Sufficiently detailed that someone can add a new row.\par
Metadata is not just for spreadsheets e.g. bibliography - where did we look, what search terms did use to we find it. Readme file - file name and description in folder.\par
Metadata should be in a different file, not a table. If you save something as a CSV, formatting is gone and you cannot have tabs.\par

\section*{16/8/2019 - 3:16pm}
\subsection*{FOAR705 Week 3 Class - Computational Thinking}

Computation thinking:\par
Decomposition - Taking a complex task and breaking it down into what needs to be done\par
Algorithm Design: articulating set of step-by-step instructions\par
Properly formatted references and bibliography exercise. Tasks to be done?
\begin{itemize}
\item Define what properly formatted and bibliography mean in terms of my thesis
\item Find required format
\item Getting the correct information in
\end{itemize}
Humans are good at collecting data, summarisation is tedious.
\section*{16/8/2019 - 3:16pm} 
\subsection*{FOAR705 Week 3 Class - Project Management Notes}
\begin{itemize}
\item Waterfall vs agile approach
\item Agile manifesto - not fetishing the process
\item Before investing time, fail early and often and test. Develop a couple of steps and test them. If they do not work, you may have to start over.
\item Do preliminary work to see if it is going to work. Respond to changes and feedback.
\item Estimate how much time each sub-task takes on Gantt chart so you can make adjustments. It is about identifying what is the critical path.
\item Kanban board can break down sub-tasks into what you need to do daily and can complement Gantt charts.
\item Kanban can limit your work in progress.
\end{itemize}

\end{document}
